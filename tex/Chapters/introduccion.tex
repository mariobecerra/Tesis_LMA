%!TEX root = ../tesis_mbc.tex

\chapter{Introducción}
\label{ch:intro}

Este primer capítulo enuncia la motivación y los objetivos de este trabajo, además de explicar los contenidos de los demás capítulos.

El objetivo de este trabajo es implementar un sistema de recomendación utilizando factorización de matrices utilizando como método de optimización descenso en gradiente estocástico. Todo se implementa desde cero, es decir, no se usa ninguna paquetería especializada en el tema. El modelo de factorización de matrices que se implementa en este trabajo fue muy utilizado entre todos los concursantes del concurso de Netflix \cite{bell2008bellkor}, aunque con distintas pequeñas variaciones. De hecho, el modelo ganador del concurso era un ensamble de muchos modelos predictores \cite{bell2007lessons}. El modelo explicado asume que se tienen calificaciones explícitas por usuario, es decir, que cada usuario da una calificación numérica a cada artículo; por ende, no es sensato utilizarlo con datos que tienen calificaciones implícitas. El alcance de este trabajo es explicar e ilustrar desde un principio los conceptos necesarios para construir e implementar un sistema de recomendación con las características ya mencionadas.

Los sistemas de recomendación se han vuelto más populares últimamente debido a un concepto: cola larga. Este concepto rompe la creencia del principio de Pareto que afirma que el 80\% de las consecuencias (ganancias) provienen del 20\% de las causas (productos). Este principio se ha usado como regla de dedo en distintas disciplinas, pero en ciertos productos no se cumple, por ejemplo, las rentas de películas en \textit{Netflix}, las cuales tienen una cola larga. En una distribución de cola larga, es menos del 80\% de las ganancias el que proviene del 20\% de los productos, esto significa que se pueden tener ganancias a partir de productos no tan populares \cite{anderson2006long}.

Con la cantidad de información que se genera hoy en día y la cantidad de servicios \textit{online} que se tienen, la cola larga (\textit{long tail}) se presenta muy a menudo. Esto porque la principal limitante de la cola larga es la escasez de los recursos. Por ejemplo, en un supermercado no se puede tener la misma cantidad de productos ofrecidos como en \textit{Ebay} porque el primero está limitado a lo que cabe en el supermercado. En servicios como \textit{Netflix} o \textit{YouTube} no se tiene esta limitante, por lo que hay muchísimos productos que se ofrecen, y para todos los productos hay algún consumidor, y en estos productos que poca gente consume se crean pequeños nichos de mercado que pueden ser explotados. De ahí nace la importancia y la necesidad de personalizar el contenido que se ofrece a los usuarios, para que puedan encontrar productos que les son de interés y que no pueden encontrar de forma sencilla debido a la enorme cantidad de opciones que se tienen. Esto le funcionó bastante bien a \textit{Netflix}, pues en un principio, el 30\% de sus rentas provenía de estrenos, comparado con el 70\% de \textit{Blockbuster}; y esto era en parte gracias al sistema de recomendación, y a que \textit{Netflix}, a diferencia de \textit{Blockbuster}, no tenía el espacio limitado para guardar películas físicamente \cite{bloombergnetflixsales}.

%Con el concepto de cola larga en mente, uno se puede imaginar por qué un sistema de recomendación puede servir para ciertos tipos de productos como \textit{Netflix} o \textit{Amazon}. Estas empresas tienen miles y miles de distintos productos que ofrecer, y muchos de ellos no son populares o conocidos por la mayoría de los usuarios, entonces, mediante un sistema de recomendación, se les pueden ofrecer productos no conocidos por ellos pero que pueden ser de su agrado. 

Un ejemplo extremo de la aplicación de los sistemas de recomendación y de la cola larga es el libro \textit{Touching the Void}. Este libro no fue muy popular cuando acababa de salir, pero años después un libro parecido llamado \textit{Into Thin Air} fue publicado. Ambos eran vendidos por \textit{Amazon}, y su sistema de recomendación encontró algunas personas que compraron ambos libros y empezó a recomendar \textit{Touching the Void} a usuarios que habían comprado o considerado comprar \textit{Into Thin Air}, y eventualmente \textit{Touching the Void} se volvió popular \cite{leskovec_mining_2014}.

% \begin{figure}
%   \centering
%     \includegraphics[width=0.5\textwidth]{LongTail.png}
%   \caption{Una distribución de cola larga (roja) con una de cola no larga (azul).}
%   \label{fig:longtailimg}
% \end{figure}

La investigación en el área de los sistemas de recomendación creció mucho a partir de 2006 cuando en octubre Netflix Inc. publicó un conjunto de datos que contenía 100 millones de calificaciones anónimas de más o menos 18 mil películas producidas por cerca de 480 mil usuarios, esto con el objetivo de retar a la comunidad científica para que crearan algoritmos que superaran el sistema que utilizaban en ese entonces, denominado \textit{Cinematch}, con una recompensa de 1 millón de dólares al primero que lo mejorara en un 10\% \cite{bell2007lessons} \cite{bennett2007netflix} \cite{netflixprize}.

En el capítulo \ref{ch:marco_teorico} se presenta la teoría matemática y estadística, los conceptos de los sistemas de recomendación, el modelo base y el modelo de factorización de matrices. En el capítulo \ref{ch:resultados} se muestra el desempeño del sistema implementado comparándolo con el modelo base con dos conjuntos de datos distintos. En el capítulo \ref{ch:conclusiones} se presentan las conclusiones y el trabajo futuro.

