%!TEX root = ../tesis_mbc.tex

\chapter{Resultados}

Los algoritmos presentados en este trabajo fueron implementados en el lenguaje para cómputo estadístico R y fueron probados con un conjunto de películas calificadas de \textit{MovieLens}, una página que ofrece recomendaciones de películas, los cuales están disponibles al público. \textit{MovieLens} es un proyecto de \textit{GroupLens}, un equipo de investigación de la Universidad de Minnesota orientado al cómputo social.

\subsection{Análisis exploratorio de datos}

\subsubsection{\textit{MovieLens}}

El conjunto de datos de \textit{MovieLens} consiste en \numprint{20000263} calificaciones de \numprint{26744} películas hechas por \numprint{138493} usuarios. Todos los usuarios habían calificado al menos 20 películas. Las calificaciones están en una escala de $0.5$ a $5$, con intervalos de medio punto, es decir, se puede poner $0.5, 1, \hdots, 4.5, 5$ como calificación \cite{harper2016movielens}. La distribución de las calificaciones se puede ver en la figura \ref{fig:ML_frec_calificaciones}; mientras que en la figura \ref{fig:ML_hist_prom_cals} se puede ver el histograma de los promedios por película.

\begin{figure}
	\centering
 	\includegraphics[width=0.8\textwidth]{frecuencia_calificaciones_MovieLens.png}
 	\caption{Frecuencia de calificaciones del conjunto de datos \textit{MovieLens}.}
 	\label{fig:ML_frec_calificaciones}
\end{figure}

\begin{figure}
	\centering
 	\includegraphics[width=0.8\textwidth]{calificacion_promedio_articulo_MovieLens.png}
 	\caption{Histograma del promedio de calificaciones por película del conjunto de datos \textit{MovieLens}.}
 	\label{fig:ML_hist_prom_cals}
\end{figure}

En la sección \ref{motivacion} se introduce el concepto de cola larga. En la figura \ref{fig:ML_long_tail} se puede ver cómo este fenómeno ocurre en los datos de \textit{MovieLens}, donde es claro que muy pocas películas tienen una gran cantidad de calificaciones, mientras que muchos artículos tienen pocas calificaciones.

\begin{figure}
	\centering
 	\includegraphics[width=0.8\textwidth]{long_tail_MovieLens.png}
 	\caption{Cola larga del conjunto de datos \textit{MovieLens}.}
 	\label{fig:ML_long_tail}
\end{figure}


\subsubsection{\textit{BookCrossing}}

El conjunto de datos de \textit{BookCrossing} consiste en \numprint{1149780} calificaciones de \numprint{271379} libros hechas por \numprint{278858} usuarios. Los datos se obtuvieron durante cuatro semanas de agosto a septiembre de 2004. Las calificaciones están en una escala de $1$ a $10$, con intervalos de un punto, es decir, se puede poner $1, 2, \hdots, 9, 10$ como calificación. Se tiene como calificación especial el $0$, el cual es solamente una calificación implícita, es decir, si leyó o no el libro \cite{ziegler2005improving}. Debido a que para usar el método presentado en este trabajo se necesitan calificaciones explícitas, se filtraron estas del conjunto de datos original. Con esta operación, quedaron \numprint{383852} calificaciones de \numprint{153683} libros hechas por \numprint{68092} usuarios. La distribución de las calificaciones filtradas se puede ver en la figura \ref{fig:BC_frec_calificaciones}; mientras que en la figura \ref{fig:BC_hist_prom_cals} se puede ver el histograma de los promedios por libro.

\begin{figure}
	\centering
 	\includegraphics[width=0.8\textwidth]{frecuencia_calificaciones_BookCrossing.png}
 	\caption{Frecuencia de calificaciones del conjunto de datos \textit{BookCrossing}.}
 	\label{fig:BC_frec_calificaciones}
\end{figure}

\begin{figure}
	\centering
 	\includegraphics[width=0.8\textwidth]{calificacion_promedio_articulo_BookCrossing.png}
 	\caption{Histograma del promedio de calificaciones por película del conjunto de datos \textit{BookCrossing}.}
 	\label{fig:BC_hist_prom_cals}
\end{figure}

En la figura \ref{fig:BC_long_tail} se puede ver la cola larga en los datos de \textit{BookCrossing}, donde nuevamente es claro que muy pocas películas tienen una gran cantidad de calificaciones, mientras que muchos artículos tienen pocas calificaciones.

\begin{figure}
	\centering
 	\includegraphics[width=0.8\textwidth]{long_tail_BookCrossing.png}
 	\caption{Cola larga del conjunto de datos \textit{BookCrossing}.}
 	\label{fig:BC_long_tail}
\end{figure}

\subsection{Comparación de modelos}

Para cada conjunto de datos se calculó un modelo base como el descrito en la sección \ref{sec:modelo_base} y un modelo de factorización como descrito en la sección \ref{sec:modelo_factorizacion}. Cada uno de los conjuntos de datos se dividió en tres subconjuntos escogidos aleatoriamente: un conjunto de entrenamiento, un conjunto de prueba y un conjunto de validación. Los parámetros de los modelos fueron estimados usando el conjunto de entrenamiento. En el caso del modelo de factorización, el conjunto de validación fue utilizado para monitorear el error en un conjunto independiente y poder utilizarlo como criterio de paro en el algoritmo de optimización y así evitar el sobreajuste. Para el modelo base, el conjunto de validación no fue utilizado.

\subsubsection{\textit{MovieLens}}

Para \textit{MovieLens}, el número de usuarios, artículos y calificaciones en cada uno de los conjuntos se puede ver en la tabla \ref{tab:ML_num_art_usu_cal}.

\begin{table}[]
	\centering
	\caption{Número de usuarios, calificaciones y artículos en los conjuntos de \textit{MovieLens}.}
	\label{tab:ML_num_art_usu_cal}
	\begin{tabular}{|l|l|l|l|}
		\hline
		Conjunto      & Número de artículos & Número de usuarios & Número de calificaciones \\ \hline
		Entrenamiento & \numprint{26247}               & \numprint{138493}             & \numprint{18029206} \\ \hline
		Validación    & \numprint{6256}                & \numprint{41483}              & \numprint{1469158} \\ \hline
		Prueba        & \numprint{2895}                & \numprint{20676}              & \numprint{501899} \\  \hline
	\end{tabular}
\end{table}

En la figura \ref{fig:ML_modelo_base_errores} se pueden ver los errores del modelo base en el conjunto de prueba, de acuerdo al valor de $\gamma$.

\begin{figure}
	\centering
 	\includegraphics[width=0.8\textwidth]{modelo_base_MovieLens.png}
 	\caption{Errores del modelo base en el conjunto de prueba de \textit{MovieLens}, de acuerdo al valor de $\gamma$.}
 	\label{fig:ML_modelo_base_errores}
\end{figure}

En la figura \ref{fig:ML_modelo_fact_errores} se pueden ver los errores de entrenamiento y de validación del modelo de factorización.

\begin{figure}
	\centering
 	\includegraphics[width=\textwidth]{errores_ent_validacion_factorizacion_MovieLens.png}
 	\caption{Errores del modelo de factorización en el conjunto de prueba de \textit{MovieLens}. LR es la tasa de aprendizaje, DL es el número de dimensiones latentes y $\lambda$ el parámetro de regularización.}
 	\label{fig:ML_modelo_fact_errores}
\end{figure}







\subsubsection{\textit{BookCrossing}}

Para \textit{BookCrossing}, el número de usuarios, artículos y calificaciones en cada uno de los conjuntos se puede ver en la tabla \ref{tab:BC_num_art_usu_cal}.

\begin{table}[]
	\centering
	\caption{Número de usuarios, calificaciones y artículos en los conjuntos de \textit{BookCrossing}.}
	\label{tab:BC_num_art_usu_cal}
	\begin{tabular}{|l|l|l|l|}
		\hline
		Conjunto      & Número de artículos & Número de usuarios & Número de calificaciones \\ \hline
		Entrenamiento & \numprint{140807}               & \numprint{64459}             & \numprint{351217} \\ \hline
		Validación    & \numprint{13072}                & \numprint{5332}              & \numprint{21224} \\ \hline
		Prueba        & \numprint{5065}                & \numprint{2286}              & \numprint{7435} \\  \hline
	\end{tabular}
\end{table}





